\documentclass[output=paper]{langscibook}
\ChapterDOI{10.5281/zenodo.15006611}
\author{Esther Jahns\orcid{}\affiliation{Carl von Ossietzky Universität Oldenburg}}
\title{“Das ist dann schon total cool zu sagen, \textit{Machanot}”: Revealing speakers' justifications for linguistic choices}
\subtitle{Revealing speakers’ justifications for linguistic choices}

\abstract{In this paper I introduce the linguistic-positioning task, a method that reveals speakers’ perspective on their own linguistic choices and the choices of others. It was developed for and will be demonstrated through the research on linguistic choices of German-speaking Jews in today's Berlin, Germany.
Like other contemporary Jewish communities (\citealt{BuninBenorHary2018}, \citealt{KahnRubin2016}), German Jews make use of what \citet[1064]{BuninBenor2008} has defined as a “distinctively Jewish linguistic repertoire”. This repertoire consists mainly of lexical items from Yiddish and Hebrew that are integrated into German.
Due to the two different donor languages, the repertoire allows for inter- and intra-speaker variation as several concepts can be expressed either in Hebrew or in Yiddish like Hebr. \textit{Mizwa}, Yidd. \textit{Mizwe} (‘good deed’). Thus, speakers as active agents can express social meaning through their meaningful choices.
It will be demonstrated how selective lexical items are used as stimuli to enhance a meta-discussion on linguistic choices. The findings show that language ideologies have the biggest impact on speakers’ choices and on their interpretation on the choices of others.
Even though the method was developed for a distinctive multilingual community, it is applicable to reveal speakers’ justifications and explanations concerning variation in other multi- and monolingual communities as well.}

\IfFileExists{../localcommands.tex}{
  \addbibresource{../localbibliography.bib}
  \usepackage{tabularx,multicol}
%\usepackage{multirow}
\usepackage{subcaption}
\usepackage{url}
\urlstyle{same}

\usepackage{datetime}
\usepackage{enumitem}
\usepackage{langsci-optional}
\usepackage{langsci-lgr}
\usepackage{langsci-branding}

\usepackage{longtable}
\usepackage{xltabular}
\usepackage[linguistics, edges]{forest}
\usepackage{pgfplots}
\pgfplotsset{compat=1.18}
\usetikzlibrary{patterns, tikzmark}
\usepackage{pgfplotstable}
\usepgfplotslibrary{colorbrewer}
\usepackage{listings}
\lstset{basicstyle=\ttfamily,keywordstyle=\normalfont,language=,breaklines=true}

\usepackage{siunitx}
\sisetup{group-digits=none, detect-all=true}

\usepackage{langsci-gb4e}

  \makeatletter
\let\thetitle\@title
\let\theauthor\@author
\makeatother

% Use this Chinese font shipped with TeX Live instead of Source Han, because
% it is more portable/leightweight. Install the "fandol" package from CTAN to
% automatically get this font.
\newfontfamily{\ChineseFandolSong}{FandolSong-Regular.otf}

  %% hyphenation points for line breaks
%% Normally, automatic hyphenation in LaTeX is very good
%% If a word is mis-hyphenated, add it to this file
%%
%% add information to TeX file before \begin{document} with:
%% %% hyphenation points for line breaks
%% Normally, automatic hyphenation in LaTeX is very good
%% If a word is mis-hyphenated, add it to this file
%%
%% add information to TeX file before \begin{document} with:
%% %% hyphenation points for line breaks
%% Normally, automatic hyphenation in LaTeX is very good
%% If a word is mis-hyphenated, add it to this file
%%
%% add information to TeX file before \begin{document} with:
%% \include{localhyphenation}
\hyphenation{
    a-na-ly-sis
    ap-proach-es
    ar-che-o-log-i-cal
    Ar-khan-gelsk
    be-schrei-ben
    Buch-holtz
    Che-lya-binsk
    con-so-nant
    dia-lect
    dia-lect-ology
    Di-a-lekt-for-schung
    Dia-lekt-for-schung
    East-pha-lian
    För-der-ung
    Ge-mein-schaft-lich-keits-ent-wür-fe
    his-tor-i-cal
    Hok-kai-do
    ja-pa-nese
    Ja-pa-nese
    Ka-go-shi-ma
    Ka-li-nin-grad
    Knja-zev
    Ma-kro-be-reich
    Ma-lay-sia
    mor-pho-log-i-cal
    Mos-cow
    Nef-te-yu-gansk
    non-mobile
    nu-cle-ar
    ös-ter-rei-chi-sche
    par-a-digm
    per-zep-ti-ons-lin-gu-is-ti-sche
    plu-ri-zen-tri-schen
    quick-ly
    Reich
    Sax-on
    Schrö-der
    sear-ching
    ste-reo-type
    strength-en-ing
    strong-est
    Stutt-gart
    su-pra-seg-men-tal
    teach-er
    to-po-gra-phy
    To-ron-to
    tra-di-tion-al
    ul-ti-mate-ly
    Um-gangs-spra-che
    Volks-kun-de
    vor-zu-stel-len
    wheth-er
    Wie-sing-er
    with-in
    Wort-at-las
}

\hyphenation{
    a-na-ly-sis
    ap-proach-es
    ar-che-o-log-i-cal
    Ar-khan-gelsk
    be-schrei-ben
    Buch-holtz
    Che-lya-binsk
    con-so-nant
    dia-lect
    dia-lect-ology
    Di-a-lekt-for-schung
    Dia-lekt-for-schung
    East-pha-lian
    För-der-ung
    Ge-mein-schaft-lich-keits-ent-wür-fe
    his-tor-i-cal
    Hok-kai-do
    ja-pa-nese
    Ja-pa-nese
    Ka-go-shi-ma
    Ka-li-nin-grad
    Knja-zev
    Ma-kro-be-reich
    Ma-lay-sia
    mor-pho-log-i-cal
    Mos-cow
    Nef-te-yu-gansk
    non-mobile
    nu-cle-ar
    ös-ter-rei-chi-sche
    par-a-digm
    per-zep-ti-ons-lin-gu-is-ti-sche
    plu-ri-zen-tri-schen
    quick-ly
    Reich
    Sax-on
    Schrö-der
    sear-ching
    ste-reo-type
    strength-en-ing
    strong-est
    Stutt-gart
    su-pra-seg-men-tal
    teach-er
    to-po-gra-phy
    To-ron-to
    tra-di-tion-al
    ul-ti-mate-ly
    Um-gangs-spra-che
    Volks-kun-de
    vor-zu-stel-len
    wheth-er
    Wie-sing-er
    with-in
    Wort-at-las
}

\hyphenation{
    a-na-ly-sis
    ap-proach-es
    ar-che-o-log-i-cal
    Ar-khan-gelsk
    be-schrei-ben
    Buch-holtz
    Che-lya-binsk
    con-so-nant
    dia-lect
    dia-lect-ology
    Di-a-lekt-for-schung
    Dia-lekt-for-schung
    East-pha-lian
    För-der-ung
    Ge-mein-schaft-lich-keits-ent-wür-fe
    his-tor-i-cal
    Hok-kai-do
    ja-pa-nese
    Ja-pa-nese
    Ka-go-shi-ma
    Ka-li-nin-grad
    Knja-zev
    Ma-kro-be-reich
    Ma-lay-sia
    mor-pho-log-i-cal
    Mos-cow
    Nef-te-yu-gansk
    non-mobile
    nu-cle-ar
    ös-ter-rei-chi-sche
    par-a-digm
    per-zep-ti-ons-lin-gu-is-ti-sche
    plu-ri-zen-tri-schen
    quick-ly
    Reich
    Sax-on
    Schrö-der
    sear-ching
    ste-reo-type
    strength-en-ing
    strong-est
    Stutt-gart
    su-pra-seg-men-tal
    teach-er
    to-po-gra-phy
    To-ron-to
    tra-di-tion-al
    ul-ti-mate-ly
    Um-gangs-spra-che
    Volks-kun-de
    vor-zu-stel-len
    wheth-er
    Wie-sing-er
    with-in
    Wort-at-las
}

  \togglepaper[1]%%chapternumber
}{}

\begin{document}
\maketitle
\label{chap:jahns}
\shorttitlerunninghead{Revealing speakers’ justifications for linguistic choices}

\section{Introduction}

Jewish speakers in contemporary Berlin have access to and make use of a “distinctive Jewish linguistic repertoire” \citep[1064]{BuninBenor2008} like Jewish speakers in other contemporary communities \citep{HaryBuninBenor2018}. This repertoire consists mainly of lexical items from Hebrew\footnote{Hebrew encompasses Biblical as well as Modern Hebrew elements. As the aim of the study is to grasp speakers’ perspective, a differentiation is not made as informants themselves generally do not differentiate between Biblical and Modern Hebrew.} and Yiddish that are integrated into German (see examples \REF{ex:jahns:1} and \REF{ex:jahns:2}).\footnote{Yiddish and Hebrew words that are part of the repertoire are used, if written, in Latin script. I decided to use, if possible, the spelling that speakers proposed or referred to \citet{Weinberg1994,Weinberg1973}.}

\ea%1
\label{ex:jahns:1}


Ich fahre auf \textit{Machane}. (Rebecca, 23:31)\footnote{All names of the interviewees have been changed.}


`I am going to a Jewish summer camp.'
\z

\ea%2
\label{ex:jahns:2}


Bist du \textit{brojges}? (Petra, 22:25)


`Are you annoyed?'

Hebrew \textit{Machane} (‘summer camp’), Yiddish \textit{brojges} (‘angry, annoyed’).
\z

As research on other contemporary communities has shown, the function of the repertoire is to index alignment to the Jewish community, but also to index subtleties of speakers’ Jewishness, i.e., to align to or distinguish from other Jewish speakers \citep[234--235]{BuninBenor2009}.

In this study I am investigating the speech pattern of Jewish speakers in 21st century Berlin.\footnote{The method described in this article is part of my study on linguistic choices and language ideologies of German-speaking Jews in contemporary Berlin (\citealt{JahnsToAppear}).} This speech pattern could be labeled a (dia)lect. For my approach and research interest the concept of the repertoire is, however, more adequate, as will be explained in more detail in the next section. The focus of this study is the variation the repertoire allows for, how speakers make use of it and how they explain their choices. Therefore, I developed a method to grasp the meaning of variation from the speakers’ perspective and to reveal their justifications and explanation for their linguistic choices and the choices of others. In this article I will describe and exemplify the method through one of the main categories that emerged from the data as affecting speakers’ choices, i.e., language ideologies towards Hebrew and Yiddish. Moreover, I will demonstrate the applicability of the methods for other multi- or monolingual communities where variation occurs.

\section{Research background}
\subsection{Multilingualism in Jewish communities in past and present}

Jewish communities have almost always been multilingual, from the 6th century BCE until today \citep{SpolskyBuninBenor2006}. However, while some aspects of this multilingualism, such as the importance of Hebrew, remained stable, others changed throughout time. Due to conquest and expulsion, Jewish communities migrated within the area of the Mediterranean, but also to other parts of Europe and to the Middle East \citep[136]{Peltz2010}. In the Diaspora, wherever Jewish communities settled, a triglossic pattern emerged. Hebrew-Aramaic\footnote{Aramaic is considered the second important language of religion as important parts of the Talmud are written in it (see \citealt[111]{Myhill2004}). Therefore, Hebrew-Aramaic is often used as a label for the sacred language of Judaism.} was the sacred language reserved for the religious domain, the language of the surrounding society was used for communication with non-Jews and eventually a third language emerged that was used for communication within the Jewish community outside the religious domain. This third language evolved due to language contact between the surrounding varieties, Hebrew-Aramaic, and other Jewish languages that the community brought with it from former areas of settlement. In the literature these vernaculars for in-group communication in Jewish diasporic communities that emerge through contact have been labelled ``Jewish languages'' (cf. \citealt{Peltz2010}, \citealt{Fishman1981}, \citealt{Gold1981}).

In the case of medieval Germany, the surrounding variety was Middle High German (MHG), or more precisely dialects of MHG. The first Jewish communities that settled in the territory of today's Germany mainly came from France and Italy and brought the Jewish languages they had spoken there with them. The contact with MHG and Hebrew-Aramaic led eventually to the emergence of Yiddish, as the Jewish language used by Jewish speakers in Ashkenaz\footnote{Hebrew name for medieval Germany}, and later on throughout western, middle and eastern Europe (\citealt{Weinreich2008}, \citealt{Jacobs2005}).

One main aspect of discussion and controversy in the field of comparative Jewish linguistics\footnote{Several other labels have been used for the linguistic field that deals with the comparison of Jewish languages, e.g., sociology of Jewish languages \citep{Fishman1981}, Jewish interlinguistics \citep{Wexler1981}.} was, however, defining the object of research and comparison, i.e., to develop unambiguous criteria for Jewish languages (e.g. \citealt{Rabin1981}, \citealt{Fishman1981}). Both parts of the label were under discussion, the Jewishness of the respective language as well as the question generally discussed in linguistics, whether a distinction can be made between language and dialect on linguistic grounds. Concerning the latter, other labels were proposed like “lect” \citep{Gold1981} or “religiolect” \citep{HaryWein2013}.

In the 21st century the differentiation between language and dialect becomes even more crucial concerning Jewish languages, as most of them range towards lesser linguistic distinctiveness in relation to the surrounding language.\footnote{This is, however, not the case in ultra-Orthodox communities where the daily vernacular is often Yiddish.} Some scholars even denied the existence or emergence of Jewish languages in the 21st century at all \citep{Myhill2004}.

Therefore, a definition as a lect could seem more adequate in the 21st century. However, Bunin Benor proposed another approach, namely the “distinctive Jewish linguistic repertoire”, which she defines as “the linguistic features Jews have access to that distinguish their speech or writing from that of local non-Jews” \citep[1064]{BuninBenor2008}. This approach changes the focus from the language (or lect) to the speech community and the (multilingual) resources to which its speakers have access. Thus, the question of (sufficient) distinctiveness is obsolete, which is not only important for Jewish speech patterns in the 21st century but also crucial when it comes to inter- and intra-speaker variation. Speakers vary not only concerning the elements from different donor languages, but also more generally concerning the number of items from the repertoire they integrate into the respective majority language. It would therefore be difficult or even impossible to determine a specific number of elements as a minimum requirement to consider the speech pattern an instance of the distinctive (dia)lect \citep[166--167]{BuninBenor2010}. Moreover, even in former times the use of Yiddish or Yiddish items was not stable across speakers, but was influenced by several categories like religiosity, geography, profession and gender \citep[37--38]{JahnsToAppear}.

The focus on the speakers allowed by the repertoire approach is also in line with 3rd wave sociolinguistics where the speaker is conceptualized as an active agent making use of all the linguistic resources she or he has access to (\citealt{Eckert2012}, \citealt{Irvine2001}, \citealt{JohnstoneEtAl2006}). Thus, speech patterns in Jewish communities in the 21st century, as well as in the past, are best described and analyzed by the repertoire approach, which I will follow here.

\subsection{Social meaning and language ideologies in the multilingual repertoire}

The variation the repertoire allows for is the prerequisite for social meaning: If two or more lexical elements denote the same referent, the choice of one of the possible variants may be used for the social positioning of the speaker. \citep[66]{JahnsToAppear} defines social meaning as “any social information speakers give about themselves or their positioning in the social landscape through linguistic means that goes beyond the denotational meaning”. This is, however, only successful if the chosen element is part of what Irvine calls a “system of distinction” (\citeyear[22]{Irvine2001}). This means the element has to be in contrast to another element in order to be salient and perceived by the hearer, and it has to be interpreted in the way the speaker intended. The interpretation depends on speakers' and hearers' social and linguistic background and experience and their contact with different groups \citep[24]{Irvine2001}. However, this system is not stable but dynamic, and changes through use, like the social meaning or index of a distinctive variant is open to interpretation and re-interpretation \citep{Eckert2008}. To trigger the intended interpretation or to understand what was meant to be expressed is therefore easier within a community that shares practices, ideas, forms of talk or history as “variation constitutes a social semiotic system capable of expressing the full range of a community’s social concerns” \citep[94]{Eckert2012}.

It must be noted that Jewish speakers in Berlin are not part of a homogeneous community. In fact, Berlin’s Jewish community is highly diverse \citep{Belkin2017}. However, all speakers do share a cultural and religious heritage which has been intertwined with two languages to which they have access due to their Jewishness, even though to different extents. The role that these two languages, Hebrew and Yiddish, have been playing in past and present for the Jewish community worldwide and for Berlin’s Jewish community in particular influences the symbolic value that speakers attribute to them, and this value in turn influences their roles. It can therefore be assumed that language ideologies will have an impact on linguistic choices, as they can be defined as “the cultural (or subcultural) system of ideas about social and linguistic relationships, together with their loading of moral and political interests” \citep[255]{Irvine1989}.

The variation or the “system of distinction” (see above) that the repertoire allows for is twofold. Firstly, it is already a choice to use an item from the repertoire instead of an item from German as the majority language. Secondly, the repertoire contains variants for several concepts or referents, mainly from the two donor languages but also from different dialects of Yiddish, like Hebrew \textit{Schabbat Schalom} and Yiddish \textit{Gut Schabbos} or \textit{Git Schabbes} (‘Shabbat greeting’). In addition to the assumed role of language ideologies concerning the choices, it is also possible that single elements are perceived as shibboleths for distinctive subgroups within the community or are used to express a certain emotional value. This is again an instance of contrast and distinction in relation to another element from the repertoire.

In sum, due to salience, distinction and contrast, the linguistic repertoire to which Jewish speakers in Berlin have access gives them multiple opportunities to vary their speech and to express social meaning. In the following sections, I will explain how I designed my study to explore the reasons for speakers’ choices from their perspective.

\section{Research design}
\subsection{Exploring the field through expert interviews}

\begin{sloppypar}
The data collection started with expert interviews with nine Jewish leaders in Berlin. Expert interviews can be used in the research design either as the main method or to explore the field by gathering expert knowledge on the topic (\citealt{MeuserNagel1991}, \citealt{BognerMenz2005}). In this case the aim was the latter, as there has not been any research on the speech patterns of contemporary Jewish communities in Germany so far. The main question to be answered through the expert interviews therefore was whether the Jewish community in Berlin actually has access to and makes use of a “distinctive Jewish linguistic repertoire” \citep[1064]{BuninBenor2008} like contemporary Jewish communities in other countries (cf. \citealt{HaryBuninBenor2018}). Subsequent aims of the exploration – if the existence of the assumed repertoire could be proven – were to get the experts’ insights on variation the repertoire allows for and an actual collection of lexical items that are part of the very repertoire.
\end{sloppypar}

Even though \citet[443]{MeuserNagel1991} state that it is the researcher who decides according to research topic and aim who can be defined as an expert, the decision about experts concerning language use within the Jewish community was not straightforward. For the aim of this study, I was looking for persons who are especially attentive concerning languages and language use or have a certain sensitivity for linguistic matters (e.g., multilinguals like Russian L1 speakers in Germany, persons working on linguistic questions without being linguists, teachers that are trained to evaluate linguistic expression of pupils) and who would come into contact with a lot of different speakers. The assumption was that group leaders might enhance the use of the repertoire for pedagogical or identity-building reasons,\footnote{This is for example the case for the website of the Jewish unity-community in Berlin and its monthly journal, where several concepts are used in Hebrew with an explanation in German following: \url{http://www.jg-berlin.org/}, 14.11.2022.} are addressed with items from the repertoire as an expression of alignment, and are in contact with other leaders/groups and might perceive differences. I tried to capture the diversity of Berlin’s Jewry by choosing leaders or representatives from different Jewish groups in order to gather a wide range of variation in the use of the assumed repertoire. The groups differed on various levels, e.g., motivations for grouping (religion, education, leisure, culture), membership categories, regularity of meetings, stability and dynamics of the groups. In addition, the experts themselves differed according to their age, gender, religiosity and L1s; they were rabbis, school and university teachers as well as leaders from different associations.

The expert interviews proved that Jewish speakers in contemporary Berlin make use of a “distinctively Jewish linguistic repertoire”, as all experts agreed upon its existence. They also stated unanimously that there is intra- and interspeaker variation, even though the opinions differed concerning the presumed reasons for it. During the interviews the experts provided items that are, according to them, part of the repertoire, and reported their use of those that have been collected in previous interviews. This collection was further expanded during the main data collection, but the main part was contributed by the experts. As of now the collection comprises 219 lexical items and formulaic sequences (see \citealt{JahnsToAppear}) and the online lexicon Judäo-Deutsches Wörterbuch \citep{Jahns2022}. This is, however, only a part of the repertoire and probably not all items that are part of the repertoire can be elicited in this manner.

However, I argue that the items mentioned by the experts are adequate stimuli to trigger explanations concerning variation and linguistic choices. It can be assumed that the experts mentioned items that were salient to them, as mentioning linguistic elements in meta-communications indicates their salience \citep[96]{Lenz2010}. \citet{Auer2014} distinguishes between three dimensions of salience: physiologically, cognitively, and sociolinguistically conditioned salience. The first one describes the pure distinction in relation to the rest of an utterance, which is probably the case for almost all items from the repertoire, as the majority stem from languages other than German. Cognitively conditioned salience refers to items that are distinguished by the hearer because s/he did not know or did not expect this item in this context. This latter means that the hearer would have used something different in this context or did not expect the speaker to make use of this item, which indicates that this is a case where inter-speaker variation happens. When salience is sociolinguistically conditioned the item stands out because it has a strong emotional value for the hearer which means it is linked with a social evaluation and indexes a certain type of speaker \citep[9--10]{Auer2014}. It can be assumed that the items the experts mentioned are salient to them regarding the second and the third dimension. The experts, who themselves share the repertoire, have been in contact with items they did not know or expect (2nd dimension) or with items that index a certain type of speaker (3rd dimension). Thus, in addition to the fact that making use of items from the repertoire is already an instance of variation (instead of using an item from majority-German), the items mentioned by the experts can be considered signs of differentiation or difference \citep{GalIrvine2019} or even shibboleths \citep{BuschSpitzmüller2021}. Therefore, these elements are adequate to trigger explanations about linguistic choices.

\subsection{Justifying perceived language use and explaining personal choices against the use of others}

The main data collection consisted of semi-structured interviews that included a task based on stimuli that were selected elements from the repertoire. The interviews were conducted with 12 Jewish speakers in Berlin and had a duration between 30--100 minutes. All interviews were recorded and transcribed with f4. The aim of the interviews was to reveal categories that affect speakers’ linguistic choices. As the analysis of the interviews was oriented towards the principles of Grounded Theory \citep{Charmaz2010}, the categories were not predefined, but were meant to emerge from the data.

Speakers were between 25 and 59 years old at the time of the interview and had different first languages (German, Russian, Polish, Swiss German, Russian and German). Eight of them considered themselves as religious (ranging from Modern Orthodox to Reform), four as secular. The interviews were conducted, when possible, in locations where the speaker would make use of the repertoire (this was for some speakers their working place) or at least in informal settings like cafés.

The interviews started with four general questions meant as ice-breakers but also as a frame-setter. The first two questions were about the Jewish community in Berlin without explicit reference to linguistic characteristics. Two further questions followed where the speaker was asked about his or her individual positioning within this community.

The main part of the interview consisted of the task which will be described in detail below. The development of the task was influenced by methods from Perceptual Dialectology (\citealt{Cramer2016a}, \citealt{Preston1999}, \citealt{Preston2010a}). The aim of Perceptual Dialectolgy (PD) is to grasp laypersons’ perspective on (dialectal) variation and where it comes from \citep[1]{Cramer2016a} which is also the aim of this study.

The draw-a-map task is the most prominent task in PD. Laypersons are asked to locate different dialects within their home-country by literally drawing them, and their boundaries, into an empty map \citep[XXXIV]{Preston1999}. My design was based mainly on the draw-a-map task as it is meant to visualize variation and difference in language use.

Yet the researched community are Jewish speakers in Berlin, which means that, like in other urban areas, boundaries between language use of different speakers or groups of speakers cannot be drawn with exact lines or isoglosses, but a visualization might look more like an intersection of sets. For these speakers – and research in urban centers in general – there are parts of the individual’s language use that overlap with the use of others and it is not possible to distinguish these uses in a geographical way like it is done in the draw-a-map task. It will be described below how I adapted this method to fit my qualitative research design. Moreover, in addition to identifying speakers’ own language use against the use of others, my task also needed to uncover the possible social meaning for the perceived variation as an expression of the individual positioning within the community.

\subsection{Selection of stimuli}

From the collected items that are part of the repertoire of German-speaking Jews in Berlin I selected 78 as stimuli for the main data collection. From these 78, I chose 50 as a minimum for each interview and added items when the interviewee was especially comfortable with the task and eager to continue. According to the principles of Grounded Theory the selection of items was also adapted in the course of the interviews when distinctive items appeared to be particularly appropriate for differentiation. This was especially the case for items from the third category (see below). The items that I selected fall into five categories.

The first category included items that have one or more variant(s) within the repertoire. This applies mainly to concepts that exist in a Hebrew and a Yiddish version in the collection like Hebrew \textit{Challa} and Yiddish \textit{Challe} (‘braided bread eaten on Shabbat and holy days’), Hebrew \textit{dawka} and Yiddish \textit{dawke} (‘on purpose; just to annoy’). But there were also concepts with more variants like Hebrew \textit{Kippa}, Yiddish \textit{Jarmulke} or \textit{Keppele} and German \textit{Käppchen} (‘skullcap’). As the use of items from the repertoire generally indexes alignment with the Jewish community, we can assume that concepts with several variants within the repertoire allow for positioning within the community.

The second category contained items that are, according to the literature or to the experts, typical for speakers from distinct networks within the community. These are items that are assumed to be known mainly by speakers who are, for example, involved in religious learning, like Hebrew \textit{Nafka Minna} (‘practical difference’); or by younger speakers who go to \textit{Machane} (‘summer camp’), like Hebrew \textit{Madrich} (‘group leader’); or by older speakers whose families have been living in Germany for generations, like Western Yiddish \textit{Barches} (‘braided bread eaten on Shabbat and holy days’).

\begin{sloppypar}
Items that triggered strong emotional reactions by some of the experts formed the third category. These were either positive reactions for items that were especially liked, e.g., \textit{brojges} (‘annoyed’), \textit{Tuches} (‘buttocks’), or negative reactions concerning items that were strongly disliked, e.g., \textit{jiddische Mame} (‘Jewish mother’), \textit{Chugist} (‘group leader’). These are examples of sociolinguistically conditioned salience. The strong reaction points either towards language ideologies or might be triggered due to the fact that the respective item is perceived as an index for distinctive speakers.
\end{sloppypar}
The fourth category consisted of items that were interesting concerning their integration into German. These are either Yiddish items that, due to the linguistic closeness of Yiddish and German, need special flagging in order to be recognized as being part of the repertoire like \textit{Mensch} (Yiddish meaning = ‘good, loyal person’; German meaning = ‘human being’), or items that are integrated through periphrastic forms, which is also a common strategy of integration in other Jewish languages (see \citealt[210--212]{Jacobs2005} for Yiddish). Examples with German auxiliaries \textit{haben} (‘to have’) and \textit{sein} (‘to be’) are \textit{Moire haben} (‘to be afraid’), \textit{Mazliach sein} (‘to succeed’).

As a fifth category I included items from other contemporary Jewish repertoires (\citealt{KlagsbrunLebenswerd2016}, \citealt{BuninBenorCohen2011}) in order to answer the question of whether a global repertoire exists or whether we find local interpretations of this very repertoire. This very small category contained local innovations from Jewish Swedish like \textit{goga} (‘synagogue’) and Yiddish items from the repertoire of American Jews that might be avoided by German-speaking Jews because of the linguistic closeness between Yiddish and German (see above) like \textit{heimisch} (‘homey’) or \textit{kwetschen} (‘to complain’)\footnote{The German verb \textit{quetschen} has, however, different semantics, meaning `to squeeze'.}.

\subsection{Linguistic-positioning task}

After the four initial questions that were described above, the main part of the interview started with the task that I developed for this study and labelled \textit{linguistic-positioning task}, which can be applied for research on linguistic variation in other groups and settings as well (see below).

Each item that was chosen for this task was written on a card and the interviewee got her or his pile of cards. The task was then to consider item after item, read each item aloud, comment on it, correct it if necessary, and finally classify it according to the speaker’s own use. The three possible categories were:

\begin{enumerate}
\item  I know this item and make use of it.
\item I know this item, but wouldn’t make use of it.
\item I don’t know this item.
\end{enumerate}

Eventually three piles of different heights emerged, visualizing each speaker’s own use (first pile) against the use of others (second and third pile). The second pile consisted of items the speaker consciously chose not to use. So, like the draw-a-map task in PD, speakers made a sort of virtual map of their language use. It is also possible to understand this task as an inversion of the Language Situations-method introduced by \citet{Wiese2020} where speakers have to imagine a specific situation and how they would describe an event in this very situation (i.e., spoken message to a friend, spoken witness statement). In the linguistic-positioning task, speakers get the linguistic element as a trigger and are encouraged to imagine a situation and/or an addressee (e.g., friend, colleague, grandparent) with whom they would make use of this very item or, if they don’t use it themselves, think of a person who would. It could therefore also be described as a focused interview where all participants were confronted with the same stimuli (cf. \citealt[195--202]{Flick2007}).

The structure of the task together with the frame-setting starting questions had several advantages and facilitated imagining contexts in which the item is or was used. First of all, the fact that the interviewee had to read the item on the card aloud was helpful, as elements from the repertoire are mainly used in spoken language.\footnote{In \citet{JahnsToAppear} it is explained in detail why items were presented in writing even though the repertoire is mainly used in spoken language.} Thus, reading, speaking and hearing the element addressed several senses and made it more likely that at least one of them would trigger the memory of a situation in which it was used by the speaker or by somebody else. Moreover, the card itself was something the speaker had to grasp and pile. This haptic part of the task, as well as the focusing on the item itself, helped to dissolve stress that might emerge in a face-to-face interview situation.

For the interviewer, the biggest advantage of the interview situation (in contrast to a questionnaire) is the opportunity to dig deeper whenever an interesting aspect or (unexpected) emotion emerged. The items that were classified in the second category (known, but not used) were of particular interest as they had the potential to express social meaning, being interpreted as signs of difference \citep{GalIrvine2019} or shibboleths \citep{BuschSpitzmüller2021}. Thus, the whole interview was a meta-linguistic discussion that was triggered by elements from the repertoire which were classified during the discussion in order to form a visualization of the speaker’s own language use in contrast to the use of others. Through the task, the speaker positioned her- or himself towards other speakers and their use \citep{Spitzmüller2013}, sometimes offering explanations that show the wish of alignment with or distinction from other speakers or speaker groups through the language use. A number of questions were prepared that the interviewer could ask whenever it seemed necessary or interesting. These questions included contexts and domains of use, (typical) users, and integration into German.

\section{Findings}
\subsection{Categories affecting speakers’ linguistic choices}

Through the process of coding the transcribed interviews, three main categories emerged that revealed to have an impact on speakers’ linguistic choices concerning the elements from the repertoire. Even though a lot of codes and subsequent categories emerged when coding the data, the following three were the most prominent and emerged through all speakers:

\begin{itemize}
\item Distinctively Jewish linguistic awareness
\item Language ideologies towards Yiddish and Hebrew
\item Intra-speaker variation
\end{itemize}

In order to demonstrate how the method that I developed works, I will give some examples from the second category and describe briefly what should be understood by the two others (for a detailed description see \citealt{JahnsToAppear}).

The category “distinctively Jewish linguistic awareness” includes all utterances that demonstrate that speakers have a distinctive awareness concerning the languages that they have access to due to their Jewishness (in this case Yiddish and Hebrew). This awareness can also be described as an advanced knowledge about etymology of the elements as well as possible variants within and across the two languages and grammar, especially concerning Hebrew. Interestingly, this awareness is very important to speakers and highly valued by them.

\ea%3
\label{ex:jahns:3}


\textit{Jesch}!\footnote{\textit{Jesch} (‘I have got’) used as ‘Yes! I’ve got it!’} […] das ist aber auch schon so 'n Codewort, dass man so 'n bisschen Hebräisch kann 


\textit{Jesch}! […] that has already become sort of a code word to show that you know a little Hebrew. [Mirjam, 08:02]
\z

Concerning the variation within the speakers, first it has to be noted that all speakers make use of items from both languages, sometimes even for the same concepts, e.g., Yidd. \textit{Tallis}, Hebr. \textit{Tallit} (‘prayer shawl’). None of them uses exclusively Yiddish or exclusively Hebrew elements, but they do use them to varying extents, with one informant trying to use as little Yiddish as possible. However, no matter how many elements they use, all speakers consider Yiddish elements to be more appropriate for informal situations with family and close friends and report using Hebrew elements predominantly in more formal situations with unknown interlocutors. Thus, the distribution of the elements from the two languages integrated into German follows a pattern that is similar to what \citet{KochOesterreicher1986} describe as language of immediacy vs. language of distance.

\subsection{Language ideologies towards Yiddish and Hebrew}

As hypothesized, language ideologies towards Hebrew and Yiddish turned out to have the strongest influence on speakers’ linguistic choices and their interpretation of utterances from others. This means that speakers explained why they would or wouldn’t use items often through (explicit or implicit) statements about the respective donor language. These explanations are in line with \citegen[193]{Silverstein1979} definition of language ideologies as a “set of beliefs about language articulated as a rationalization or justification of perceived language structure and use”. Interestingly, speakers started their evaluation of an item quite often with a remark concerning its language of origin, even though this was not part of the task. However, this knowledge, that I labeled distinctively Jewish linguistic awareness (see above), was an indispensable prerequisite for using languages and their symbolic value as an explanation. Knowing about the origin of a lexical element and being able to differentiate between the two donor languages was the basis for justifying one’s language use through beliefs about the respective language of origin and/or typical speakers who make use of it.

The explanations and justifications uttered by the speakers are not expressions of one single ideology, but form clusters or ``language ideological assemblages'' \citep[134]{Kroskrity2018}. Generally speaking, it can be said that the ideologies towards Yiddish reflect the more emotional, but also conflicting, relation speakers have with this language. It is conflicting also due to its changing role from a vernacular for the majority of European Jewry to a marker of ultra-Orthodoxy. This emotional relationship with Yiddish has the effect that speakers often try to describe their connection to Yiddish or its role by distinctive concepts or even lexical elements from Yiddish, which I then used as in-vivo codes, e.g., nostalgia or \textit{Stetl} (‘little town’). However, to what extent and in what direction the respective ideology influences the speaker’s choices depends on her or his individual biography and actual positioning. The two following quotes from my data are examples for the ideology that I labelled “Nostalgia for the \textit{Stetl}” and they demonstrate very clearly how the same ideology can be interpreted in two directions:

\eanoraggedright%4
\label{ex:jahns:4}
\begin{otherlanguage}{ngerman}
...und Jiddisch ist sowas, was ich so 'n bisschen mehr mit so Tradition und meiner Herkunft verbinde, weil ich weiß meine Großeltern und Eltern kamen ja auch aus Osteuropa, aus Rumänien und äh mein Vater ist auch mit Jiddisch aufgewachsen. Und dann weiß man irgendwie, wenn man diese Wörter benutzt, dann ist so irgendwie so dieses Alte, dieses Traditionelle, das so 'n bisschen weiterlebt. Das hat sowas ähm Nostalgisches und auch was so nja, vielleicht sowas Symbolisches, Bedeutungsvolles. Deswegen find ich das schon ganz schön dann solche Begriffe auch mal dann mitzunehmen. 
\end{otherlanguage}


… and Yiddish is something that is for me more connected with tradition and my origins, because I know that my parents and grandparents came from Eastern Europe, from Romania and um my father was raised with Yiddish. And then you know when you use these words, then in a way it’s like something old, something traditional, lives on a little bit. There’s something um nostalgic about it and also something, well maybe symbolic, meaningful. That is why I do like sometimes to pick up these terms. [Rebecca, 1:06:52]
\z

For Rebecca, keeping up some Yiddish words and integrating them into German indexes a link to her past and triggers nostalgia. Petra, on the other hand, completely rejects the use of Yiddish as an index for backwardness. She, like others, uses the Yiddish noun \textit{Stetl} (‘little town’) as a metonymy for the traditional and religious Jewish life in Eastern Europe. She considers this world vanished and it has no relevance for her kind of Jewishness. According to her, Hebrew, on the contrary, is an adequate expression also for a secular Jewish identity.

\eanoraggedright%5
\label{ex:jahns:5}
\begin{otherlanguage}{ngerman}
Ich würde sagen, dass ich mich im Hebräischen viel wohler fühle als im Jiddischen, weil ähm, ich glaube, dass das Jiddische eben dann aufgesetzt wäre, dass man sich 'ne Welt aneignet, die es nicht mehr gibt. […] Also jedenfalls nicht in den jüdischen Kreisen in denen ich mich bewege, ne?! Das ist keine \textit{Stetl}-Welt und dementsprechend hat das Jiddische da auch keine Relevanz und keinen Platz. Das Hebräische aber schon, das moderne Hebräisch ist für mich ein Ausdruck des Jüdischseins, äh, der mir einfach entspricht, weil der eben auch, äh, säkular geht. 
\end{otherlanguage}

I would say, that I am much more comfortable with Hebrew than with Yiddish, cause hum, I think that Yiddish would then be artificial, that you try to take possession of a world that does not exist anymore. […] At least not within the Jewish circles that I am moving in, see?! That is no \textit{Stetl}{}-world and therefore, there is no place for Yiddish in it and Yiddish has no relevance in it. But Hebrew has, modern Hebrew is for me an expression of Jewishness, hum, that fits me, because it, um, it works also in a secular way. [Petra, 42:04]
\z

The quote from Petra also demonstrates another ideology towards Yiddish that is very prominent across my data, namely authenticity \citep[167--174]{JahnsToAppear}. Like other speakers, Petra seems to accept the use of Yiddish items mainly from an “authentic” speaker, who is a person that has been raised with Yiddish. Taking over Yiddish elements or learning the language later in life is perceived as inauthentic or, as Petra puts it, \textit{aufgesetzt} (‘artificial’).

The ideologies towards Hebrew are generally more stable, both through time and across speakers, which shows the high prestige Hebrew has been enjoying by almost all speakers. Generally, speakers describe the relation towards Hebrew more through reasoning and do not use lexical items or emotions to do so. The next quote demonstrates the prestige of Hebrew as the language of Judaism and underlines its utility, while Yiddish is considered as less important and only interesting from a cultural or scientific perspective.

\ea%6
\label{ex:jahns:6}
\begin{otherlanguage}{ngerman}
Also das Hebräische ist halt einfach die Sprache des Judentums. So. Und die Tora ist halt eben die Heilige Schrift auf Hebräisch. So. Und ja, und Jiddisch ist natürlich total schön und total interessant usw., aber es ist halt eigentlich, glaube ich mehr aus so 'ner ja, soziologischen Perspektive, kulturell-soziologischen Perspektive interessant zu untersuchen und linguistisch sicherlich auch, aber im, jetzt für die für des Judentum an sich, für den Bestand, sag ich mal, des Judentums ist es jetzt nicht so wichtig, glaub ich, Jiddisch zu können. Also da ist es viel wichtiger Hebräisch zu können, um das Judentum sozusagen aufrecht zu erhalten, weil das Hebräische nun mal einfach die, die Sprache des Judentums ist, inhaltlich, und immer sein wird. Und Jiddisch, kann halt sein, dass es irgendwie in 50 Jahren keiner mehr kennt oder so, ja? 
\end{otherlanguage}

Well, Hebrew is just simply the language of Judaism. So. And the Torah is of course the holy book in Hebrew. So. And, yes, and Yiddish is of course really nice and really interesting and so on, but it is, however, I think, more from a sociological perspective, cultural-sociological perspective interesting to investigate and sure enough also linguistically, but in, now for Judaism as such, it is, I think, not so important to know Yiddish. Thus, it’s more important to know Hebrew to maintain Judaism, so to speak, because Hebrew is just simply the, the language of Judaism, in terms of content, and always will be. And Yiddish, it might be, that nobody will know it in 50 years or so, yes? [Julia, 1:17:34]
\z

\begin{sloppypar}
This high prestige goes along with the ideology of prescriptivism, which means that it is considered important to use Hebrew words in the “correct” way. This includes the Hebrew stress pattern which differs from German and especially the plural morphemes. When I asked about the integration of Hebrew nouns like \textit{Kippa} (‘skullcap’) and \textit{Machane} (‘Jewish summer camp’), speakers unanimously agreed upon the fact that these nouns should be used with the Hebrew plural inflection, i.e., \textit{Kippot} and \textit{Machanot}, when integrated into German.
\end{sloppypar}
In the next quote the speaker describes that he would correct children at a Jewish summer camp who would add the German plural morpheme -\textit{s} to the Hebrew noun \textit{Madrichim} (‘Youth supervisors’) that is already in the Hebrew form (sg. \textit{Madrich} – pl. \textit{Madrichim}).

\ea%7
\label{ex:jahns:7}


Wenn 'nen Kind kommt und sagt ‚Wo sind die Madrichims?', sag ich \textit{{}‚Madrichim!'}. 


If a child comes to me and says `Where are the Madrichims?’, I say ‘\textit{Madrichim}!’. [Aaron, 37:05]
\z

In quote \REF{ex:jahns:8} the interviewer explicitly asks whether the adding of a German plural morpheme -\textit{s} would be a possible variant:

\ea%8
\label{ex:jahns:8}


Interviewer:  Gibts auch jemand der Machanes sagt? Oder könnte \hspace*{20mm} man das sagen?


Julia:  \hspace*{11mm} Also, der wäre sozusagen ziemlich außen als Idiot. Quasi, \hspace*{20mm} das ist dann schon total cool zu sagen, so                  \textit{Machanot}. Da hat \hspace*{20mm} man ja Ahnung, wie der Plural gebildet wird im \hspace*{20mm} Hebräischen. […] Jaa. Ne, also das, da würde man sofort \hspace*{20mm} korrigiert werden. 


Interviewer:   Are there people who say Machanes? Or is it possible to say \hspace*{20mm} it?


Julia:  \hspace*{11mm} Well, this person would be rather outed as an idiot. I mean, \hspace*{20mm} it’s pretty cool then to say, like \textit{Machanot}. You obviously \hspace*{20mm} have a clue then how the plural is formed in Hebrew. […] \hspace*{20mm} Yeah. No really, you would definitely be corrected. [21:50]
\z

The quote shows that the prestige the language enjoys within the community is transmitted to the user of the language, or in this case, to the user of elements from the language. The transmission of prestige does, however, apply only if the element is used in the “correct” way. Speakers who can’t show their distinctively Jewish linguistic awareness are stigmatized as outsiders or newcomers.

\section{Conclusion and applicability}

In this study I have shown how Jewish speakers in Berlin make sense of the variation within the linguistic repertoire to which they have access. The variation consists mainly of speakers’ choices between lexical elements from Yiddish and Hebrew or between dialectal variants of Yiddish, but it encompasses also the quantity of items used. The linguistic-positioning task reveals speakers’ perspectives on their linguistic choices leading to categories that affect these choices.

In this study, the importance of knowing about, and differentiating between, the two main donor languages; intra-speaker variation due to context-awareness; and, most importantly, language ideologies towards Hebrew and Yiddish, were revealed as being the most important categories that affect speakers' choices.

Hebrew elements are considered more neutral but prestigious variants that should be used in a “correct” way. Hebrew allows for the expression of Jewishness also for secular speakers. Yiddish indexes a vanished world that some have nostalgia for, while others reject it. Its use triggers strong emotions in both directions and is reserved for the “authentic” speaker.

However, other categories emerged as well from my data which would allow to answer slightly different questions using the same material, i.e., the transcribed interviews. I would like to give some examples of what these research questions or different perspectives on the data might look like.

First of all, it would be interesting to take a much closer look at individual elements and their use by the interviewees. A detailed statistical analysis of the distribution into the three piles and its comparison across speakers might lead to clusters of use or correlations between speakers sharing characteristics, like age, L1, and religiosity. In addition, the findings from the three piles could be used to get insights into a core vocabulary that is used almost by everyone and thus is used for indexing alignment to the Jewish community as a whole instead of foregrounding subtleties of different kinds of Jewishness.

The task could also give insights into the salience of distinctive variants. In this study some items were not perceived by all speakers as being part of a distinctively Jewish repertoire. This was the case with elements that are cognates in Yiddish and German and therefore seem not to be salient when integrated into a German sentence. An example is \textit{heimisch} (‘homey’) even though contexts of use seem to be slightly different in the two languages, or \textit{Macher} (‘active person’). It is no surprise that speakers with German as their L1 did not consider them part of the Jewish repertoire, while speakers with Russian as their L1 or who lived outside Germany for a longer time considered themja Yiddish and even expressed an especially positive emotion towards it.

Insights on ongoing language change concerning distinctive elements is another possible outcome of this task. Several items from the repertoire were either classified as known, but not used or not known, which could indicate that they are no longer in use, but have been overheard from the generation of parents or grandparents, e.g., Yiddish \textit{Barches} (‘braided bread eaten on Shabbat and holy days’).

\begin{sloppypar}
I have demonstrated that the linguistic-positioning task captures, from a speaker’s perspective, justifications and explanations for lexical variation in the multilingual community of Jewish speakers in today's Berlin.
\end{sloppypar}

The task, however, is also applicable for multiple other sets of data and other linguistic areas. It is an effective tool to reveal reasons for linguistic choices, including ideologies and prestige of involved languages or varieties. It could also be used to find out about the salience of distinctive elements and whether they are used or understood as shibboleths for distinctive groups. This could, in consequence, give further insights into speakers’ positioning in relation to these presumed groups, whether they wish to align to or want to distinguish from them. Moreover, the task is not restricted to lexical variation but can also be applied to pronunciation and syntactic patterns.

The task does not necessarily need to be applied to multilingual groups of speakers but can be used to research monolingual speakers and any kind of variation in their speech, e.g., dialectal variation, youth language, or language of distinct networks variation.

\sloppy\printbibliography[heading=subbibliography,notkeyword=this]
\end{document}
